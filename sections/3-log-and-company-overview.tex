\documentclass[../main.tex]{subfiles}
\begin{document}
\section{Company Overview}
En beskrivelse af virksomheden, forretningsgrundlag og organisation 

Netcompany is Denmark's biggest IT consultancy firm. They deliver IT solutions to big businesses and governments, mostly in northern Europe.
They have more than 7000 employees. It was founded in 1999. 

Netcompany is broadly divided in to two types of teams. Project delivery teams that build and implement new things, and long-term service teams that support long-term applications

There is no hard divide between software development and operating the software, DevOps. Teams are expected to handle both for their applications. 

I am an intern at the APS (Application Services) division of Netcompany.
Originally, APS were only tasked with maintaining existing software, that either came from the outside or that the consultancy division wanted to handover.
Sometimes there are new projects in APS, but that is usually a smaller part of a larger project that is being maintained.  

Netcompany has a Methodology that is called Agile with control. Agile aims to avoid planning to far ahead, since you can not know what will happen far in to the future.
Since many projects are done together with state actors, who demand long term planning, Netcompany tries to do agile, while also satisfying client demands. 
\section{Log Book}
\subsection{Uge 1 - Oplæring - \textbf{Intro day}}

\textit{2025-02-03 Mon}

Today was intro day. From 08.30 to 14.30 there were practical things, getting access to systems, taking photos, and getting a computer and phone. I still don't have my phone. From 14.30 until 16.30 I met with my team and made introductions. I don't have access to the relevant repo yet.

\subsubsection{\textbf{Setup access problems}}

\textit{2025-02-04 Tue 13:34}

I got access to repo today, but I can't access the relevant Docker files so I can't run the program. I will work on, \href{https://nationaltforsoegsoverblik.dk/}{NFO} and my first task will be to work on an Admin Panel, so Sys Admin can add and edit FAQ's, short guides, available sicknesses, text descriptions and activity logs.

It took 2 hours in the morning for us to realize that the problem is that I don't have access to the docker files, and I am still waiting for IT to approve my access. It is 13.34 now. I will read the todo-list, P0120 - Løsningsbeskrivelse til systemadministrator modul, through and if I have time, I will start looking at the Angular code and start learning Angular

\subsubsection{\textbf{Setup version problems}}

\textit{2025-02-04 Tue 16:22}

Around half an hour my setup started working. I had to use Java 11 for main product, Java 17 for IDP fake login, and node version was also wrong. I had 20 and you need 18 or below.

The setup was very complicated also with many xml and properties files that needed to be copied from a certificate repo to the main repo so everything would work for authentication.

This took a lot of time, but Christian, one of the software engineers helped me. He was very patient. I then helped my colleague Marcus Vilhemsen with setting everything up, because he had the same problems that I had. It was difficult remembering the one-hour long walkthrough of setting it up from Christian, but with some help from git diff I managed to remember.

\subsubsection{\textbf{Locked out again}}

\textit{2025-02-06 Thu 10:20}

Today when I got into work my setup had stopped working. The verification part of the system wants to pull a docker file, but it doesn't have access to the file anymore. I think my access has been revoked. I have written IT service, so now I must wait for them.

\subsubsection{Meeting with old new friends}

\textit{2025-02-06 Thu 10:25}

Now we are going to have a meeting with all new people at 10.30 

\subsection{Uge 2 - Admin Panel \href{https://www.nationaltforsoegsoverblik.dk/}{nafo.dk}}
\subsubsection{\textbf{A few productive days and then a wall}}

\textit{2025-02-11 Tue}

Friday and Monday was very productive, I have gotten a lot done one the FAQ admin panel editor. Yesterday at around 16 I got problems with docker again, and have spent two hours yesterday, and two hours two day trying to get things to work. In frustrating moments like this, it is important to remember an old saying.

\begin{quote}
\emph{This, too, shall pass.}
\end{quote}

Just don't let the frustration consume me is probably one of the most important lessons I can learn, in IT and life.

\subsubsection{\textbf{Got some work done}}

\textit{2025-02-12 Wed}

Today I got a lot of work done on an admin panel for \href{https://www.nationaltforsoegsoverblik.dk/}{nafo.dk}. I have created angular components for all the routes in the admin panel.

There is going to be editors for the following things on the site:

Descriptions and texts on the site

\begin{itemize}
\item
  FAQ
\item
  Disease hierarchy
\item
  Quickguides
\item
  Trials
\item
  Users
\item
  Locations(regions)
\end{itemize}

I am not sure if I should use routing for each of them of if should just make it a SPA. The senior developer mentoring me, Simon, has said I can decide, and that either works, but I think it is often best to use separate routes.

\subsubsection{\textbf{Finished FAQ admin editor}}

\textit{2025-02-13 Thu} I have finished work on the FAQ admin editor, but I am still waiting for the code to be reviewed, and approved. The backend took longer than I thought. There is also always a prioritization to make between doing things formally correct and between doing them in a reasonable time-span. One example of this was the following:

The FAQ\_item table in PostgreSQL has the following schema:

\begin{verbatim}
    
CREATE SEQUENCE IF NOT EXISTS faq_item_seq;
  
CREATE TABLE IF NOT EXISTS "faq_item" ( 
  "id"       bigint NOT NULL DEFAULT NEXTVAL('faq_item_seq'),
  "question" text   NOT NULL,
  "answer"   text   NOT NULL,
  "ordering" bigint NOT NULL,
    PRIMARY KEY ("id")
);

\end{verbatim}

The interesting line is the following, which can be written in two distinct ways:

\begin{verbatim}
"ordering" bigint NOT NULL,
vs
"ordering" bigint NOT NULL UNIQUE,
\end{verbatim}

Unique makes sense since if you a bunch of items that are supposed to be ordered and re-ordered, then no two items should be in the same spot. This sounds great until you try to re-order the items, and you realize that re-ordering items, when no two items can ever have the same value means you can not just modify the ordering column.

There are three alternatives solving this while keeping the UNIQUE constraint

\begin{enumerate}
\def\labelenumi{\arabic{enumi}.}
\item You can move them item.ordering to a very high offset and then re-order the items below. Should work in practice but will in theory not work if \verb|offset < length(incoming_faq_items)|. 

\item  Create a temporary table where you have id, and ordering, and create new order there, and then move the ordering atomically to the original ordering column. This works, but the Java code we work in, really doesn't like when you use the DDL(Data Definition Language) instead of the DML(Data Manipulation Language), \href{https://stackoverflow.com/questions/2578194/what-are-ddl-and-dml}{StackOverflow, DDL vs DML}. It complicated things, and lead me to also abandoned this idea.
\item Create a temporary column and then set temp\_col equal to ordering similar to this: \verb|UPDATE faq_item SET ordering = temp_col|
\end{enumerate}


I spent 20 minutes on this last solution, and then realized, my time is wasted here. This won't really matter. Let's get something working, and then solve this problem if it ever occurs. Since the frontend always sees all \verb|faq_items|, we will always get re-order requests for all of the data in the table, so even if we have no UNIQUE, constraint, we should very rarely or never encounter a but with this.

If a bug occurs, then it will be solved by the end-user reordering the items once again.\strut

\subsection{Uge 3 - Faq Admin Text description editor} 

\textbf{How to talk during office drama}

\textit{2025-02-19 Wed}

\textbf{If somebody wants to do something stupid or get me to do
something stupid}

Don't start an unnecessary conflict. Instead say:

``Yes, I understand, interesting. Can you get the boss up-to-date on
this?''

\textbf{If somebody has done something stupid that I want to revert} Let's say someone has forgotten to add access control to the backend controllers. Many people don't like pointing out that they have done a mistake.

Don't say:

''Hey you forgot this. This is dangerous and insecure. ``

Instead say:

``Do you want me to add \emph{@RolesAllowed} annotations to the PUT, POST, and DELETE methods in \emph{FAQResource.java}? If I remember correctly, it should be \emph{SYSTEM\_ADMIN} who has access.''

Or the more generic version:

``Do you want me to fill out the rest of the details in the assignment? I think it should be something like A, B and C.''

I have been a bit frustrated recently by some things and tried to find a way of saying.

The problem was someone changing the code I had submitted and then making some mistakes. But it all worked out in the end. 

\subsection{Uge 4 - Faq Admin Text description editor} \textbf{Finishing second to last part of admin panel}

\textit{2025-02-26 Wed}

Work is going slow but steadly. The problem is that I always think I am
close to being done, but there always come up new things that are not
working as they should. 

5 - Select to multiselect specialization - \textbf{String to
List\textless String\textgreater{}}

\textit{2025-03-04 Tue}

There was a select box where physician can select the specialization of the study of the people doing the study. Before you could only pick one, e.g. Oncology. Now the client wants to enable picking multiple, e.g. Oncology and Pediatrician and Surgeon. Sounds very easy.

I thought it would be easy. It required changes to around 30-40 files, 700 lines of code changed and adding three new tables to the database. Enterprise legacy software is really something else, but now it
works. 

6 - Email templates, change PI navn, improved modal share - 1 on 1
meeting with my team manager Pernille

\textit{2025-03-06 Thu}

DIGST is short for Digitaliserings Styrelsen. My girlfriend is a what in danish is called a "jurist", which is similar to the english lawyer, and works at DIGST. She writes the danish AI law proposal implementation from the recent EU directives.

We have just recently got a contract for DIGST, where we implement some web things for them. I was worried if this is a conflict of interest. Pernille told me to not worried.

I don't get very much feedback from the one very good developer on the project, because he is very busy with other projects. The junior developers have started asking me for advice, because I often know a lot about some niche programming things, and have worked with front end a lot. Am I living up to expectations? I mean I have made commits who modify 60-70 files and adds 700 LOC(lines of code) and delete 200 LOC, and I get almost no response. My intuition is that no feedback is good feedback. This is not kindergarten, so we don't get unneccesary praise, but it is slightly difficult to know, so I asked here how they think it is going.

I told her I have don't have a clear feeling for how it is going.

I asked, how do you and the other Team Manager, Søren Olsen, think it is going?

For context I want to describe Søren. Søren, is a silent, stern and serious man. I have heard from others that he can scold or berate people from his team if he is not happy with their performance, or if they have done something stupid. He is usually their in our 15 minute morning meetings with NFO project. He really know how put people in their place.

There is never any doubt about wheter he is happy or not with your performance. Fortunately he has gone easy on me so far.

Pernille let me know that she has talked with Søren and he had said that he was impressed and happy about the work I've
done.

That really calmed me done, and made me stress about things less. \\

\subsection{Uge ??? \textbf{Work Offer}}

\textit{2025-04-02}

Netcompany has asked if I want to keep working here as a
``studentermedarbejder'' after I finish my internship. This made me very
glad, and I think I will say yes. I hope it will not be to much work
next semester, but I am happy about being here and my colleagues. \\

\subsection{Uge ???? \textbf{Working on SOCBS(Social- og Boligstyrelsen) project}}

\textit{2025-04-08}

I have worked a lot on something called VIAS. It is the sagstyringssystem that \textit{SOCBS} uses. There is a new law about Greenland and some additional funding that should be added to the things Socialstyrelsen do there, and that means some of the code must change.

This code is a lot older than my previous project. I think some of it is at least 15-20 years. 

\subsection{Uge ??? \textbf{Internal systems have crashed part 1}}

\textit{2025-04-18}

All the cases that are registered by \textit{SOCBS} employees, also get registered to a system called F2. These last few days that has not been working. 

This is a big emergency for them, since this means that their employees can't work until we get this fixed.
We develop the internal case handling system, but F2 is developed by a separate company and we have no contact with them.
This makes everything a more cumbersome. Our system tries to register the cases at F2, but we get an error when trying to do that. 
This indicates that the error is on their side, and not ours. The problem could be almost anything.
Even though we suspect that we are not at fault for the error, the client has asked us to look in to it, and see if there is anything we can do.

\subsection{Uge ??? \textbf{SOCBS internal systems have crashed part 2}}

\textit{2025-04-25}

The problem has been solved, and it was the third parties, fault as suspected. Now there are a lot of users that have registered cases with \textit{SOCBS} that have not been registered properly. 
My job currently is to go through the logs, looking at 100 000's of thousands of error messages, find the relevant ones find the ID of the customer, and then find their phone numbers, so \textit{SOCBS} can contact them.

\subsection{Uge ??? \textbf{Writing solution proposal}}

\textit{2025-05-09 Fri}

Niklas, a senior developer, gave me the task of writing a løsningsbeskrivelse.

The problem \textit{SOCBS} wants solved is that at the end of each year, they must manually move many budget posts from one year to the next in their internal systems. If they have a 100.000 dkk surplus, in one of their budgets it has to be moved by hand.
This is error prone and time-consuming. They want a way of automatically moving the budgets when the new year starts.

My task is to find a way of solving this and explain it to the client in a document, so that they can accept, decline, or ask for more information.

The document is a standard template we use, and it has 6 headings.

\end{document}